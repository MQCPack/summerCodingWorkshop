\LoadClass[homework,duemidnight]{hph}
\setcoursename{Chem 212}
\setactivitynumber{3}
\setduedate{December 11, 2018}
%
\begin{document}
\makeheaderfooter{}
\maketitle

%
% Altered particle in a box using Mathematica.
\begin{problem1}
Consider a one-dimensional particle in a box where the potential is
$\infty$
for $x < {}0$ and $x > L$. In the range from $0$ to $L$, let the
potential energy be given by $V\left(x\right) = m x$. Using $h=m=L=1$,
write a Mathematica notebook that solves for the eigenfunctions and
eigenvalues of the first five states of this system. Use the linear variational
method to carry out this numerical problem. The standard one-dimensional
particle-in-a-box eigenfunctions should be used as your basis set.

\begin{subproblem1}
How does the ground state energy vary as a function of the number of basis
functions? To explore this point, begin by using the two lowest energy
states of the standard particle-in-a-box system as your set of basis
functions. Then, use the first three states of the standard
particle-in-a-box as your basis set. Follow this by numerical tests using
four, five, six, seven, eight, nine, and ten states. Plot the ground state
energy as a function of the number of basis functions used.
\end{subproblem1}

\begin{subproblem1}
Repeat the previous experiment with the potential function changed to
$V\left(x\right) = 10 m x$.
\end{subproblem1}

\begin{subproblem1}
Plot the ground state wavefunction for the previous two problems using an
appropriately converged basis set size. Comment on the effect of the added
potential on the shape of the ground state wavefunction.
\end{subproblem1}

\end{problem1}
%
\end{document}
